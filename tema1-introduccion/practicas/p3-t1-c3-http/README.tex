\PassOptionsToPackage{unicode=true}{hyperref} % options for packages loaded elsewhere
\PassOptionsToPackage{hyphens}{url}
%
\documentclass[]{article}
\usepackage{lmodern}
\usepackage{amssymb,amsmath}
\usepackage{ifxetex,ifluatex}
\usepackage{fixltx2e} % provides \textsubscript
\ifnum 0\ifxetex 1\fi\ifluatex 1\fi=0 % if pdftex
  \usepackage[T1]{fontenc}
  \usepackage[utf8]{inputenc}
  \usepackage{textcomp} % provides euro and other symbols
\else % if luatex or xelatex
  \usepackage{unicode-math}
  \defaultfontfeatures{Ligatures=TeX,Scale=MatchLowercase}
\fi
% use upquote if available, for straight quotes in verbatim environments
\IfFileExists{upquote.sty}{\usepackage{upquote}}{}
% use microtype if available
\IfFileExists{microtype.sty}{%
\usepackage[]{microtype}
\UseMicrotypeSet[protrusion]{basicmath} % disable protrusion for tt fonts
}{}
\IfFileExists{parskip.sty}{%
\usepackage{parskip}
}{% else
\setlength{\parindent}{0pt}
\setlength{\parskip}{6pt plus 2pt minus 1pt}
}
\usepackage{hyperref}
\hypersetup{
            pdfborder={0 0 0},
            breaklinks=true}
\urlstyle{same}  % don't use monospace font for urls
\usepackage{color}
\usepackage{fancyvrb}
\newcommand{\VerbBar}{|}
\newcommand{\VERB}{\Verb[commandchars=\\\{\}]}
\DefineVerbatimEnvironment{Highlighting}{Verbatim}{commandchars=\\\{\}}
% Add ',fontsize=\small' for more characters per line
\newenvironment{Shaded}{}{}
\newcommand{\AlertTok}[1]{\textcolor[rgb]{1.00,0.00,0.00}{\textbf{#1}}}
\newcommand{\AnnotationTok}[1]{\textcolor[rgb]{0.38,0.63,0.69}{\textbf{\textit{#1}}}}
\newcommand{\AttributeTok}[1]{\textcolor[rgb]{0.49,0.56,0.16}{#1}}
\newcommand{\BaseNTok}[1]{\textcolor[rgb]{0.25,0.63,0.44}{#1}}
\newcommand{\BuiltInTok}[1]{#1}
\newcommand{\CharTok}[1]{\textcolor[rgb]{0.25,0.44,0.63}{#1}}
\newcommand{\CommentTok}[1]{\textcolor[rgb]{0.38,0.63,0.69}{\textit{#1}}}
\newcommand{\CommentVarTok}[1]{\textcolor[rgb]{0.38,0.63,0.69}{\textbf{\textit{#1}}}}
\newcommand{\ConstantTok}[1]{\textcolor[rgb]{0.53,0.00,0.00}{#1}}
\newcommand{\ControlFlowTok}[1]{\textcolor[rgb]{0.00,0.44,0.13}{\textbf{#1}}}
\newcommand{\DataTypeTok}[1]{\textcolor[rgb]{0.56,0.13,0.00}{#1}}
\newcommand{\DecValTok}[1]{\textcolor[rgb]{0.25,0.63,0.44}{#1}}
\newcommand{\DocumentationTok}[1]{\textcolor[rgb]{0.73,0.13,0.13}{\textit{#1}}}
\newcommand{\ErrorTok}[1]{\textcolor[rgb]{1.00,0.00,0.00}{\textbf{#1}}}
\newcommand{\ExtensionTok}[1]{#1}
\newcommand{\FloatTok}[1]{\textcolor[rgb]{0.25,0.63,0.44}{#1}}
\newcommand{\FunctionTok}[1]{\textcolor[rgb]{0.02,0.16,0.49}{#1}}
\newcommand{\ImportTok}[1]{#1}
\newcommand{\InformationTok}[1]{\textcolor[rgb]{0.38,0.63,0.69}{\textbf{\textit{#1}}}}
\newcommand{\KeywordTok}[1]{\textcolor[rgb]{0.00,0.44,0.13}{\textbf{#1}}}
\newcommand{\NormalTok}[1]{#1}
\newcommand{\OperatorTok}[1]{\textcolor[rgb]{0.40,0.40,0.40}{#1}}
\newcommand{\OtherTok}[1]{\textcolor[rgb]{0.00,0.44,0.13}{#1}}
\newcommand{\PreprocessorTok}[1]{\textcolor[rgb]{0.74,0.48,0.00}{#1}}
\newcommand{\RegionMarkerTok}[1]{#1}
\newcommand{\SpecialCharTok}[1]{\textcolor[rgb]{0.25,0.44,0.63}{#1}}
\newcommand{\SpecialStringTok}[1]{\textcolor[rgb]{0.73,0.40,0.53}{#1}}
\newcommand{\StringTok}[1]{\textcolor[rgb]{0.25,0.44,0.63}{#1}}
\newcommand{\VariableTok}[1]{\textcolor[rgb]{0.10,0.09,0.49}{#1}}
\newcommand{\VerbatimStringTok}[1]{\textcolor[rgb]{0.25,0.44,0.63}{#1}}
\newcommand{\WarningTok}[1]{\textcolor[rgb]{0.38,0.63,0.69}{\textbf{\textit{#1}}}}
\setlength{\emergencystretch}{3em}  % prevent overfull lines
\providecommand{\tightlist}{%
  \setlength{\itemsep}{0pt}\setlength{\parskip}{0pt}}
\setcounter{secnumdepth}{0}
% Redefines (sub)paragraphs to behave more like sections
\ifx\paragraph\undefined\else
\let\oldparagraph\paragraph
\renewcommand{\paragraph}[1]{\oldparagraph{#1}\mbox{}}
\fi
\ifx\subparagraph\undefined\else
\let\oldsubparagraph\subparagraph
\renewcommand{\subparagraph}[1]{\oldsubparagraph{#1}\mbox{}}
\fi

% set default figure placement to htbp
\makeatletter
\def\fps@figure{htbp}
\makeatother


\date{}

\begin{document}

\hypertarget{descripciuxf3n-de-la-pruxe1ctica-p3-t1-c3-http}{%
\section{Descripción de la práctica
p3-t1-c3-http}\label{descripciuxf3n-de-la-pruxe1ctica-p3-t1-c3-http}}

\begin{enumerate}
\def\labelenumi{\arabic{enumi}.}
\tightlist
\item
  Siguiendo el capítulo 20 \emph{Node.JS} de la segunda edición del
  libro Eloquent JavaScript (EJS) escriba sus propios apuntes con
  ejemplos y realice los ejercicios que se indican a continuación

  \begin{itemize}
  \tightlist
  \item
    \href{https://eloquentjavascript.net/2nd_edition/20_node.html}{Eloquent
    JS: Chapter 20 HTTP}
  \end{itemize}
\item
  Realice el ejercicio \emph{Creating Directories}
\end{enumerate}

\begin{itemize}
\tightlist
\item
  Though the \texttt{DELETE} method is wired up to delete directories
  (using \texttt{fs.rmdir}), the file server currently does not provide
  any way to create a directory. Add support for a method
  \texttt{MKCOL}, which should create a directory by calling
  \texttt{fs.mkdir}
\end{itemize}

\begin{enumerate}
\def\labelenumi{\arabic{enumi}.}
\setcounter{enumi}{3}
\tightlist
\item
  Instale \href{https://insomnia.rest/}{insomia} o
  \href{https://www.getpostman.com/}{postman} para usarlo como cliente
  de prueba.
\item
  Genere documentación para su código usando algunas de las herramientas
  que aparecen en la sección recursos
\item
  Escriba un gulpfile con tareas usando \texttt{curl} para probar el
  comportamiento del servidor con los diferentes requests. Aquí tiene un
  ejemplo (incompleto) en gulp 3.9:
\end{enumerate}

\begin{Shaded}
\begin{Highlighting}[]
\KeywordTok{var}\NormalTok{ gulp }\OperatorTok{=} \AttributeTok{require}\NormalTok{(}\StringTok{"gulp"}\NormalTok{)}\OperatorTok{;}
\KeywordTok{var}\NormalTok{ shell }\OperatorTok{=} \AttributeTok{require}\NormalTok{(}\StringTok{"gulp-shell"}\NormalTok{)}\OperatorTok{;}

\VariableTok{gulp}\NormalTok{.}\AttributeTok{task}\NormalTok{(}\StringTok{"pre-install"}\OperatorTok{,} \VariableTok{shell}\NormalTok{.}\AttributeTok{task}\NormalTok{([}
      \StringTok{"npm i -g gulp static-server"}\OperatorTok{,}
      \StringTok{"npm install -g nodemon"}\OperatorTok{,}
      \StringTok{"npm install -g gulp-shell"}
\NormalTok{]))}\OperatorTok{;}

\VariableTok{gulp}\NormalTok{.}\AttributeTok{task}\NormalTok{(}\StringTok{"serve"}\OperatorTok{,} \VariableTok{shell}\NormalTok{.}\AttributeTok{task}\NormalTok{(}\StringTok{"nodemon server.js"}\NormalTok{))}\OperatorTok{;}

\VariableTok{gulp}\NormalTok{.}\AttributeTok{task}\NormalTok{(}\StringTok{"lint"}\OperatorTok{,} \VariableTok{shell}\NormalTok{.}\AttributeTok{task}\NormalTok{(}\StringTok{"jshint *.js **/*.js"}\NormalTok{))}\OperatorTok{;}

\VariableTok{gulp}\NormalTok{.}\AttributeTok{task}\NormalTok{(}\StringTok{"get"}\OperatorTok{,} \VariableTok{shell}\NormalTok{.}\AttributeTok{task}\NormalTok{(}\StringTok{"curl -v https://localhost:8000/file.txt"}\NormalTok{))}\OperatorTok{;}
\VariableTok{gulp}\NormalTok{.}\AttributeTok{task}\NormalTok{(}\StringTok{"put"}\OperatorTok{,} \VariableTok{shell}\NormalTok{.}\AttributeTok{task}\NormalTok{(}\StringTok{"curl -v -X PUT -d 'Bye world!' https://localhost:8000/file.txt"}\NormalTok{))}\OperatorTok{;}

\end{Highlighting}
\end{Shaded}

\begin{enumerate}
\def\labelenumi{\arabic{enumi}.}
\setcounter{enumi}{5}
\tightlist
\item
  Entregue los enlaces al repositorio en GitHub
\end{enumerate}

\hypertarget{recursos}{%
\subsection{Recursos}\label{recursos}}

\begin{itemize}
\tightlist
\item
  \href{https://eloquentjavascript.net/2nd_edition/20_node.html}{Eloquent
  JS 2nd Edition: Chapter 20 HTTP}
\item
  \href{https://github.com/ULL-ESIT-MII-CA-1718/nodejs-KevMCh}{Repo con
  las soluciones K.} (No disponible ahora)
\item
  \href{https://github.com/ULL-ESIT-MII-CA-1718/ejs-chapter20-node-js}{Repo
  con las soluciones C.} (No disponiblei ahora)
\item
  \href{https://youtu.be/nuw48-u3Yrg}{How to Develop Web Application
  using pure Node.js (HTTP GET and POST, HTTP Server)} Vídeo en Youtube.
  2015
\item
  \href{https://nodejs.org/en/docs/guides/anatomy-of-an-http-transaction/}{Anatomy
  of an HTTP Transaction}
\item
  Documentación:

  \begin{itemize}
  \tightlist
  \item
    \href{https://documentation.js.org/}{documentation.js},
  \item
    \href{https://www.npmjs.com/package/jsdoc}{jsdoc},
  \item
    \href{https://jashkenas.github.io/docco\%60}{docco}
  \end{itemize}
\item
  Gulp

  \begin{itemize}
  \tightlist
  \item
    Véase la sección
    \href{https://casianorodriguezleon.gitbooks.io/ull-esit-1617/apuntes/gulp/}{Gulp}
    de los apuntes
  \item
    \href{https://gulpjs.com/docs/en/getting-started/quick-start}{gulp
    quick start}
  \item
    \href{https://gulpjs.org/getting-started.html}{gulp getting started}
  \end{itemize}
\item
  Diseño

  \begin{itemize}
  \tightlist
  \item
    \href{https://casianorodriguezleon.gitbooks.io/ull-esit-1617/content/apuntes/patterns/codesmell.html}{Apuntes:
    Code Smells}
  \item
    \href{https://casianorodriguezleon.gitbooks.io/ull-esit-1617/content/apuntes/patterns/designprinciples.html}{Principios
    de Diseño}
  \item
    \href{https://casianorodriguezleon.gitbooks.io/ull-esit-1617/content/apuntes/patterns/}{Patrones
    de Diseño}
  \item
    \href{https://casianorodriguezleon.gitbooks.io/ull-esit-1617/content/apuntes/patterns/strategypattern.html}{Strategy
    Pattern}
  \end{itemize}
\item
  \texttt{/Users/casiano/local/src/javascript/eloquent-javascript/chapter20-node-js/}
  (recurso para el profesor)
\end{itemize}

\hypertarget{reto}{%
\subsection{Reto}\label{reto}}

\begin{itemize}
\tightlist
\item
  \href{reto.md}{Reto para la práctica}
\end{itemize}

\end{document}
